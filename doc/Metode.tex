% -*- TeX:Rnw -*-
% ----------------------------------------------------------------
% .R Sweave file  ************************************************
% ----------------------------------------------------------------
%%
% \VignetteIndexEntry{}
% \VignetteDepends{}
% \VignettePackage{}
%\documentclass[a4paper,12pt]{article}
\usepackage[slovene]{babel}
\usepackage[utf8]{inputenc} %% must be here for Sweave encoding check
\newcommand{\SVNRevision}{$ $Rev: 3 $ $}
%\newcommand{\SVNDate}{$ $Date:: 2009-02-2#$ $}
\newcommand{\SVNId}{$ $Id: program.Rnw 3 2009-02-22 17:36:08Z ABlejec $ $}
%\usepackage{babel}
%\input{abpkgB}
%\input{abpkg}
\input{abBeam}
\input{abcmd}
%\input{abpage}
\usepackage{pgf,pgfarrows,pgfnodes,pgfautomata,pgfheaps,pgfshade}
\usepackage{amsmath,amssymb}
\usepackage{colortbl}
\usepackage{Sweave}
\input{mysweaveB}
\newcommand{\BV}{}
\newcommand{\EV}{}
\newcommand{\myemph}[1]{{\color{Sgreen} \textit{#1}}}
\input{./figs/SwPres-concordance}
%\SweaveOpts{echo=false}
%\usepackage{lmodern}
%\input{abfont}
%\SweaveOpts{keep.source=true}
% ----------------------------------------------------------------
\title{Izbira in uporaba statističnih metod\\Seminar}
\author{A. Blejec}
%\address{}%
%\email{}%
%
%\thanks{}%
%\subjclass{}%
%\keywords{}%

\date{11. in 12. april 2013}%
%\dedicatory{}%
%\commby{}%
\begin{document}
\mode<article> {\maketitle}
\mode<presentation> {\frame{\titlepage}}
\tableofcontents
% ----------------------------------------------------------------
\begin{abstract}
 
\end{abstract}
% -------------------------------------------------------------
%% Sweave settings for includegraphics default plot size (Sweave default is 0.8)
%% notice this must be after begin{document}
% \setkeys{Gin}{width=0.9\textwidth}
\setkeys{Gin}{width=0.7\textwidth}
% ----------------------------------------------------------------

% -*- TeX:Rnw -*-
% ----------------------------------------------------------------
% .R Sweave part file  *******************************************
%
\usepackage[utf8]{inputenc} %% must be here for Sweave encoding check
% ----------------------------------------------------------------
\section{Članki}
%% |--------------------->>>>>>
\begin{frame}[fragile]
\frametitle{Uporaba statistike v znanstvenih člankih}
\begin{itemize}
  \item Introduction
  \item \emph{Methods}
  \item \emph{Results}
  \item Discussion
\end{itemize}
\end{frame}
%% <<<<<<---------------------|
% -*- TeX:Rnw -*-
% ----------------------------------------------------------------
% .R Sweave part file  *******************************************
%
\usepackage[utf8]{inputenc} %% must be here for Sweave encoding check

\section[Merjenje]{Merske lestvice}

%% |--------------------->>>>>>
\begin{frame}[fragile]
\frametitle{Merske lestvice \hfill (Stevens, Science, 1946)}

\begin{block}{Opisna \hfill \emph{atributivna}}
  \begin{itemize}
    \item Imenska \hfill \emph{nominalna}
    \item Urejenostna \hfill \emph{ordinalna}
  \end{itemize}
\end{block}

\begin{block}{Številska \hfill \emph{numerična}}
  \begin{itemize}
    \item Razmična \hfill \emph{intervalna}
    \item Razmernostna \hfill \emph{racionalna}
  \end{itemize}
\end{block}
\end{frame}
%% <<<<<<---------------------|

%% |--------------------->>>>>>
\begin{frame}[fragile]
\frametitle{Imenska lestvica}
\begin{block}{Primeri}

\begin{itemize}
  \item Spol: moški, ženski ; M / F
  \item Barva: rdeča, rumena, modra, zelena
  \item Habitat: gozd, močvirje, travnik
  \item Temperatura: prijetna, neprijetna
\end{itemize}
\end{block}
\begin{block}{Lastnosti}

\begin{itemize}
  \item Oznake omogočajo razlikovanje stanj
  \item Nobene naravne urejenosti (je pred/je za)
  \item Lahko štejemo
  \item Ne moremo računati npr. povprečja !!!
  \end{itemize}
\end{block}

\end{frame}
%% <<<<<<---------------------|

%% |--------------------->>>>>>
\begin{frame}[fragile]
\frametitle{Imenska lestvica}
\begin{block}{Dihotomna lestvica}
Temperatura
\begin{description}
  \item[0] Neprijetna
  \item[1] Prijetna
\end{description}
\vspace{6pt}
\hrule
\vspace{6pt}
\emph{Temperatura je prijetna} (FALSE=0 / TRUE = 1)

\end{block}
\begin{Schunk}
\begin{Soutput}
n = 10
\end{Soutput}
\begin{Soutput}
P P P N N P P N N P
\end{Soutput}
\begin{Soutput}
1 1 1 0 0 1 1 0 0 1
\end{Soutput}
\begin{Soutput}
1+1+1+0+0+1+1+0+0+1 = 6 / 10
\end{Soutput}
\begin{Soutput}
Povprečje: 0.6
\end{Soutput}
\begin{Soutput}
60% enot ima lastnost iz trditve (prijetno)
\end{Soutput}
\end{Schunk}

\end{frame}
%% <<<<<<---------------------|



% ----------------------------------------------------------------





% ----------------------------------------------------------------
\bibliographystyle{amsplain}
\bibliography{ab-general}
\end{document}
% ----------------------------------------------------------------
