% -*- TeX:Rnw -*-
% ----------------------------------------------------------------
% .R Sweave file  ************************************************
% ----------------------------------------------------------------
%%
% \VignetteIndexEntry{}
% \VignetteDepends{}
% \VignettePackage{}
\documentclass[a4paper,12pt]{article}
\usepackage[slovene]{babel}
\usepackage[utf8]{inputenc} %% must be here for Sweave encoding check
\newcommand{\SVNRevision}{$ $Rev: 39 $ $}
%\newcommand{\SVNDate}{$ $Date:: 2010-05-26 #$ $}
\newcommand{\SVNId}{$ $Id: Sweave.Rnw 39 2010-05-25 22:22:19Z ABlejec $ $}
\input{abpkg}
\input{abcmd}
\input{abpage}
\usepackage{pgf,pgfarrows,pgfnodes,pgfautomata,pgfheaps,pgfshade}
\usepackage{amsmath,amssymb}
\usepackage{colortbl}
\usepackage{Sweave}
\input{mysweave}
\input{./figs/bla-concordance}
%\SweaveOpts{echo=false}
\setkeys{Gin}{width=0.8\textwidth}  % set graphicx parameter
%\SweaveOpts{keep.source=true}
\usepackage{lmodern}
\input{abfont}

% ----------------------------------------------------------------
\begin{document}
%% Sweave settings for includegraphics default plot size (Sweave default is 0.8)
%% notice this must be after begin{document}
%%% \setkeys{Gin}{width=0.9\textwidth}
% ----------------------------------------------------------------
\title{Izbira in uporaba statističnih metod\\Seminar}
\author{A. Blejec}
%\address{}%
%\email{}%
%
%\thanks{}%
%\subjclass{}%
%\keywords{}%

\date{11. in 12. april 2013}%
%\dedicatory{}%
%\commby{}%
\maketitle
% ----------------------------------------------------------------
%\begin{abstract}
%
%\end{abstract}
% ----------------------------------------------------------------
%\tableofcontents

\section*{Program}

\begin{itemize*}
  \item Statistika v znanstvenih člankih
  \item Merske lestvice
  \item Ocenjevanje parametrov, intervali zaupanja
  \item Logika testiranja hipotez
  \item Parametrični, neparametrični in permutacijski testi
  \item Primerjava dveh skupin
  \item Odvisnost med pojavi\\ (regresija, ANOVA, kontingenca, logistična regresija)
  \item Linearni modeli
\end{itemize*}
\section*{Urnik}
Seminar bo potekal 11. aprila (10h - 13h) in 12. aprila (9h - 12h) v prostorih MBP.


% ----------------------------------------------------------------
%\bibliographystyle{chicago}
%\addcontentsline{toc}{section}{\refname}
%\bibliography{ab-general}
%--------------------------------------------------------------

%\clearpage
%\appendix
%\phantomsection\addcontentsline{toc}{section}{\appendixname}
%\section{\R\ funkcije}
%\input{}

\clearpage
\section*{SessionInfo}
{\small
Windows 7 x64 (build 7601) Service Pack 1 \begin{itemize}\raggedright
  \item R version 2.15.1 (2012-06-22), \verb|x86_64-pc-mingw32|
  \item Locale: \verb|LC_COLLATE=Slovenian_Slovenia.1250|, \verb|LC_CTYPE=Slovenian_Slovenia.1250|, \verb|LC_MONETARY=Slovenian_Slovenia.1250|, \verb|LC_NUMERIC=C|, \verb|LC_TIME=Slovenian_Slovenia.1250|
  \item Base packages: base, datasets, graphics, grDevices,
    methods, splines, stats, utils
  \item Other packages: Hmisc~3.10-1, patchDVI~1.9,
    survival~2.36-14
  \item Loaded via a namespace (and not attached):
    cluster~1.14.2, grid~2.15.1, lattice~0.20-6, tools~2.15.1
\end{itemize}Project path:\verb' D:/_Y/R/metode '\\Main file :\verb' ../doc/Program.Rnw '
\subsection*{View as vignette}
Project files can be viewed by pasting this code to \R\ console:\\
\begin{Schunk}
\begin{Sinput}
> projectName <-"metode";  mainFile <-"Program"
\end{Sinput}
\end{Schunk}
\begin{Schunk}
\begin{Sinput}
> commandArgs()
> library(tkWidgets)
> openPDF(file.path(dirname(getwd()),"doc",
+ paste(mainFile,"PDF",sep=".")))
> viewVignette("viewVignette", projectName, #
+ file.path("../doc",paste(mainFile,"Rnw",sep=".")))
> #
\end{Sinput}
\end{Schunk}

\vfill \hrule \vspace{3pt} \footnotesize{
%Revision \SVNId\hfill (c) A. Blejec%\input{../_COPYRIGHT.}
%\SVNRevision ~/~ \SVNDate
\noindent
\texttt{Git Revision: \gitCommitterUnixDate \gitAbbrevHash{} (\gitCommitterDate)} \hfill \copyright A. Blejec\\
\texttt{ \gitReferences} \hfill \verb'../doc/Program.Rnw'\\

}



\end{document}
% ----------------------------------------------------------------
